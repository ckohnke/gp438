\documentclass[11pt]{article}
\usepackage[margin=1in]{geometry}
\usepackage{timeline}
\usepackage{graphicx}
\usepackage{color}
\usepackage{float}
\usepackage{hyperref}
\usepackage{mathtools}

\usepackage[style=authoryear, backend=bibtex]{biblatex}
\addbibresource{final_bib.bib}
\nocite{*} 

\usepackage{listings}
\lstset{ %
language=JAVA,                % choose the language of the code
basicstyle=\footnotesize,       % the size of the fonts that are used for the code
numbers=left,                   % where to put the line-numbers
numberstyle=\footnotesize,      % the size of the fonts that are used for the line-numbers
stepnumber=1,                   % the step between two line-numbers. If it is 1 each line will be numbered
numbersep=5pt,                  % how far the line-numbers are from the code
backgroundcolor=\color{white},  % choose the background color. You must add \usepackage{color}
showspaces=false,               % show spaces adding particular underscores
showstringspaces=false,         % underline spaces within strings
showtabs=false,                 % show tabs within strings adding particular underscores
frame=single,           % adds a frame around the code
tabsize=2,          % sets default tabsize to 2 spaces
captionpos=b,           % sets the caption-position to bottom
breaklines=true,        % sets automatic line breaking
breakatwhitespace=false,    % sets if automatic breaks should only happen at whitespace
escapeinside={\%*}{*)}          % if you want to add a comment within your code
}

\author{Advisor/Client: Dave Hale \\ Team Members: Colton Kohnke}
\title{GPGN438 Senior Design: \\ Exploratory Seismic Data Analysis in the Field}
\date{December 9th, 2013}

\begin{document}
\maketitle
\newpage

\tableofcontents
\listoffigures
\listoftables
\newpage

\section{Executive Summary}

Seismic surveys take a significant amount of time and money to complete correctly. This project aims to develop software for the Colorado School of Mines (CSM) to aid with viewing seismic survey data and geometry in the field. The software will also be able to help catch errors in the field. It will also serve as a teaching tool for the students of the Colorado School of Mines Geophysics Field Camp to help them better understand seismic surveys. \\

The software will accomplish this by reading the SEGD files in real time from the recording truck along with the observation notes and the flag locations from a GPS unit. This data will then be plotted in map and section view in order to give a snapshot of the seismic survey. Basic processing can then be done quickly including gain, amplitude balancing, lowpass filtering, normal-moveout correction and surface-wave attenuation to create a dynamic stacked section. \\

The project will be completed by May 2014 and will be used as a replacement for ProMAX in the Summer 2014 Geophysics Field Camp. 

\section{Problem Statement}

This project seeks to develop software in JAVA that aids with the exploration of seismic data in the field. The survey exploration includes the interactive display of survey geometry and seismographs. This software will also be able to perform simple processing tasks in the field. The program will serve as a bridge from the seismic crew to the students of Field Camp and serve as a teaching tool for the student's understanding of the seismic survey.

\section{Introduction \& Background}

Geophysical data collection is an expensive operation that costs both time and money. In the field, it is in the data collector's best interest to make sure that the data collected is good quality. \\

ProMAX is the current system used in the field to view the seismic data coming in from the recording station. This system, while good for in depth processing, is not robust enough to quickly display survey data and interactively display the survey geometry. The CSM Field Camp is unique in that they surveys are not conventional, that is, the surveyors are trying parameters or setups that aren't typical of a corporate survey. \\

There can also be a lack of understanding between the students and the seismic survey with regards to how the data is collected in the field. This software will seek to help the students at Field Camp better understand the seismic survey while they are in the field. The software can also be used as a teaching tool to the students in Field Camp to help them better understand the aspects of a seismic surveys.

\section{Deliverables to Client}

The deliverables of this project include two main items. The first is a program written in JAVA that accomplishes the design objectives outlined in the next section. The second deliverable will be a documentation of the code in both standard notation and as a PDF "how-to" style guide.

\section{Design Objectives}
\subsection{Interactive Display of Survey Geometry}

The first objective is to be able to import station (flag) locations. This needs to be accomplished from a variety of GPS sources including CSV, Excel, GPX and tab-delimited text files. The GPS coordinates need to then be converted to UTM for easier processing. Once these stations have been imported and converted, they need to have the option of being exported so the conversion does not need to be applied again. The suggested columns are listed in Table \ref{TAB:GPS}. Once the conversion is applied, the next step is to plot the locations in map view.

\begin{table}[H]
\caption{GPS Export Spreadsheet Fields}
\centering
\begin{tabular}{ c | c | c | c}
  \hline                  
  StationID & UTM Easting & UTM Northing & Elevation \\
  \hline
\end{tabular}
\label{TAB:GPS}
\end{table}

The second objective is to be able to compute source and receiver locations. This is done for each valid shot FFID (Field File ID) by using the observer files to determine the source station number, live recording channels and receiver station numbers. The station locations are then used to find the location of the source and the elevation is read from a USGS Digital Elevation Map. All of this data is then stored into a spreadsheet file with columns denoted in Table \ref{TAB:SRL}. Some FFIDs correspond to bad shots and will need to be ignored dynamically by the program routines. \\

\begin{table}[H]
\caption{Source-Receiver Locations Spreadsheet Fields}
\centering
\begin{tabular}{ l | l | l | l | l | l | l}
  \hline                  
  FFID & SEGD Filename & Source & StationID & Channel Number & SourceXYZ & ReceiverXYZ \\
  \hline
\end{tabular}
\label{TAB:SRL}
\end{table}

Once the data has been extracted from the observation files, the program needs to be able to plot source, receiver and midpoint locations in map view. In map view, there needs to be able to plot a piecewise curve that represents the seismic line. The program also needs to be able to plot elevation profiles along the source, receiver and midpoint lines. Finally, there needs to be a slider to select and show points corresponding to each FFID that will provide a graphical history of the seismic survey and help to catch mistakes. 

\subsection{Interactive Display of Seismograms}

The first step to displaying the seismograms is to convert the SEGD files (bytes) to an array of an array of floats (2D flaot array). Once this is accomplished, the seismograms need to be interactively displayed by using the FFID slider. This will display all seismograms for a particular FFID. \\

Another method of display will be to display all seismograms within a circle that the user draws in map view. \\

Two more sliders, called the min-max sliders, that are populated with the minimum and maximum offsets will be implemented. As these sliders move, the program will plot all seismograms with offsets within the range of the min-max sliders. \\

All of this is useless unless the data can actually be seen. Therefore, interactive controls for gain, amplitude balancing and lowpass filtering (Butterworth) need to be implemented and the results plotted. This will be done by using sliders, or interactive graphical menus.

\subsection{Simple Processing}

In the field, there is often the need to do basic processing in order to see a better picture of what is happening in the subsurface. Tools for basic processing of the data will be included in the developed software. The guaranteed processing tools will be for surface-wave attenuation and normal-moveout correction, but more are being investigated.

\subsection{Other Requirements}

Speed is a top priority when working in the field. Decisions are typically made quickly in order to move the survey along at a reasonable rate. If this software is fast enough, it can be used as a supplement to those decisions. \\

The user-interface must also be easy to use. The layout of the controls must be logical and follow a sequential order. These different features must be documented in the end-user documentation and be consistent in their implementation. 

\section{Decision-making \& Assessment of Alternative Approaches}

The decisions during the main software development section of this project deal with efficiency. It is possible that choices in the field need to be made very quickly and as a result, the software needs to be able to perform operations efficiently to keep pace. If the software is called to display a seismic trace, it needs to be able to find the data and display it in the least amount of operations. Similarly, if an operation is applied to a set of data, it should be applied in the least amount of compute time. \\

The software is not currently optimized for speed because that would slow down development. However, specific places in the software have been marked for efficiency improvements during the testing phase. One area that has been marked for optimization is the storage of the data from the SEGD files. Currently, shots are stored as a JAVA ArrayList and is traversed linearly. This causes an O($n^2$) search time where $n$ is the length of the array. A much faster implementation would be a Binary Search Tree which has O($n$) traversal time. Another option is implementing a hast table, which returns a value in near constant time, O($1$). \\

The decisions during the later stages will revolve around end user usability. These choices will include aesthetic placement of tools in the software to create a logical progression of tasks for the end user. The decisions in this section will also revolve around how to lay out the documentation for the end user. These decisions will be made during the final stages of the project and will likely involve bringing in fresh eyes to test the documented work flows. \\

\section{Design Solution}

The tool that is being used in this design project is simulation. Each method that is written is tested by calling the method on the 2013 Field Camp data. These simulations provide the developer with real data to test the software against and with speed/efficiency benchmarks to try to improve. \\

In addition, before a feature is integrated into the larger code, it is tested against itself for errors and correct implementation. This speeds up the development process because it limits the area where bugs can occur at any given time.

\section{Implementation Plan}
\subsection{Safety}

The main safety concern of this project is the ergonomics of the developer. This project does not require field work or manual labor in the traditional sense. However, it does require a significant amount of time to be spent coding at a computer which may cause irritation of the wrist or forearm. The developer will mitigate the risk of encountering this problem by taking breaks, stretching and using correct posture while at the workstation.

\subsection{Timeline}

\begin{figure}[H]

\begin{timeline}{2013}{2015}{200}{300}
  \categorylabel{Senior Design Timeline}{60}
  
  \MonthAndYearEvent{9}{2013}{Project start}
  \MonthAndYearEvent{10}{2013}{JAVA code}
  \MonthAndYearEvent{11}{2013}{Sercel SEGD reading and plotting Jython code converted to JAVA.}
  \MonthAndYearEvent{11}{2013}{Demo with interactive plots created}
  \MonthAndYearEvent{1}{2014}{Core features implemented}
  \MonthAndYearEvent{1}{2014}{Alpha testing starts}
  \MonthAndYearEvent{1}{2014}{Code optimization begins}
  \MonthAndYearEvent{3}{2014}{Beta testing begins}
  \MonthAndYearEvent{3}{2014}{Documentation writing begins}
  \MonthAndYearEvent{5}{2014}{Final product ready for use at CSM Field Camp 2014}
  
\end{timeline}
\caption{Senior Design Project Timeline}
\end{figure}

\subsection{Division of Responsibility \& Effort}

The project team is composed of a single developer, so the whole responsibility is on the only member of the team. 

\subsection{Budget 1: Actual}

No budget is required for the actual project as the project does not require any traveling or other expenditures outside the realm of a standard class at CSM.

\subsection{Budget 2: Professional}

Ideally to get a working product to the client, the following budget would be used in a professional setting. 

\begin{table}[H]
\caption{The Professional Budget}
\begin{tabular}{ l | l | l || l | l}
  \hline                        
  Item & Base Cost & Quantity & Total & Notes \\ \hline
  Developers & \$20.00/hr & 360 & \$7,200 & 2 Devs for 4 work weeks.\\
  Continued Support & \$15.00/hr & 260 & \$3,900 & 1 Dev at 5 hours/week for 1 year.\\ \hline
  Base Total & & & \$11,100 & \\
  Overshoot & & & \$1,110 & 10\% addition\\ \hline
  Total & & & \$12,210 & \\
  \hline  
\end{tabular} 
\end{table}

\section{Implementation}
\subsection{Data Acquisition}

The data used to help develop the software is the 2012 CSM Geophysics Field Camp seismic data provided by Dr. Dave Hale. No further data acquisition is planned for this project. 

\subsection{Data Reduction, Analysis, Interpretation, Integration}

This project revolves more around the display of field data than the analysis and active interpretation of the data. Certain reductions will be implemented to the data in an interactive way. These reductions include gain, lowpass filtering and amplitude balancing. The software will also serve as a hub for a place to view all the seismic data for a survey and integrate it together to make plots of the seismic line, such as a brute stack.

\subsection{Error Analysis}

The error analysis of this project will be related to the processing and display of imported data. These processes include making sure tools like the gain and filtering are correct, and the plotting of the data is accurate. \\

Currently the code for the gaining and filtering the data has been written, but it has not been tested against a known working algorithm. The shot plotting function has been tested against the known working Jython code and has been confirmed to work as expected.

\subsection{Results}

The results so far of the project have been the implementation of software that imports GPS points from Lat/Lon and transforms them into UTM coordinates before plotting them in map view. The software also imports SEGD files from the Sercel format and plots the nearest shot interactively as the user moves a mouse around the map view screen. The GPS and SEGD data can be imported and exported to files for easier imports later. Basic tools for manipulating the data have also been implemented, including gaining the data and passing the data through a lowpass filter.

\subsection{Discussion and Conclusions}

So far, everything has been running well and the setbacks have been mitigated by meetings with the advising professor.  \\

The main pitfall the team has faced on this project has been time. As a single developer, I have not been able to dedicate enough time to this project and I should be much farther along at this point in time. The second half of the project will be more strenuous, but the results will come faster because my own coding ability has increased throughout this project. 

\section{Recommendations for Future Work}

The next step for the project is to read the observation reports and finish implementing the main core features to create an alpha product. From there, code optimization can begin to occur and more efficient testing can begin. \\

After the software has been optimized for speed, it then needs to be optimized for usability. This will allow for the end-user to efficiently use the software.

\section{References}

\printbibliography

\newpage
\section{Resumes}

%% LaTeX file for resume 
% This file uses the resume document class (res.cls)

\documentclass[margin]{res}
%\usepackage{helvetica} % uses helvetica postscript font (download helvetica.sty)
%\usepackage{newcent}   % uses new century schoolbook postscript font  
\topmargin=-0.6in  % start text higher on the page
\setlength{\textheight}{11in} % increase text height to fit resume on 1 page
\usepackage{color}

\begin{document}  
\name{COLTON KOHNKE}
\address{1221 Illinois Street Apt. 2E \\ Golden, CO 80401 \\ (360) 813-2795 \\ ckohnke@mines.edu }

\begin{resume}                        

\section{EDUCATION} \textbf{Colorado School of Mines, Golden, CO} \hfill B.S. May 2014 \\
                Geophysics \& Geophysical Engineering \hfill CUM GPA 3.46 \\
                \textcolor{white}{...} \hfill Major GPA 3.61 \\
                \textbf{Professional Organizations}
                \begin{itemize}
                 \item Society of Exploration Geophysicists Student Member (SEG)
                 \item European Association of Geoscientists and Engineers (EAGE)
                \end{itemize}

\section{WORK \\ EXPERIENCE}
		  %\begin{tabular}{p{3in} r} 
			\textbf{Chevron of North America, Bakersfield, CA} \hfill Summer 2013 \\
			Upstream Technical Computer G{\&}G Support Intern
		  		\begin{itemize}
		  		\item Designed, developed, documented and deployed a new error reporting system to ensure well data integrity between 						multiple databases.
		  		\item Assessed opportunity of a new database/software system for seismic metadata to increase efficiency and safety while decreasing cost.
		  		\item Worked with Data Analyst and Asset Development teams to standardize well status symbols for use within the San Jaoquin Valley.
		  		\end{itemize}
		  		
			\textbf{Ventyx, Greenwood Village, CO} \hfill Summer 2012 - June 2013 \\
			MineScape Engineer Intern
                  %\end{tabular}	
                   \begin{itemize} % \item[] prevents a bullet from appearing
                    \item Wrote, debugged, tested and integrated code into MineScape software.
                    \item Tested the new MineScape 5.4 release.
                    \item Traveled and assisted clients with overburden stratigraphic modeling and \\geophysical interpretation on an 							open-pit coal mine.
                    \item Wrote and updated documentation for consumer utilization.
				   \end{itemize} 

\section{ENGINEERING \\ \& TECHNICAL \\ SKILLS} 
		\begin{itemize}
		\item MineScape, MineScape Programming Language, Geolog, OpenWorks, Petrel.
		\item JAVA/C++, MatLab, Python, R, LaTeX, and Mathematica.
		\item Geological Lab Experience, Geophysical Field Methods, Geophysical Survey \\ Design, GPS, Field and Office Safety, Well-Log Analysis.
		\item AutoCAD, SolidWorks; Microsoft Excel, Word, Powerpoint, Publisher.
		\item Linux/UNIX, Windows, and Mac OS.
		%\item Finite Element Analysis, Systems, Computer Aided Data Acquisition, Graphical and Written Design Proposals, Oral Presentation, Technical Writing, Formal Proposal Writing, Budgeting, and Scheduling.
		\end{itemize}
		
\section{PROJECT \\ EXPERIENCE}
		  %\begin{itemize}
			 %\item
            \textbf{Senior Design: Exploratory Analysis of Field Seismic Data} - Senior capstone project written in JAVA to quickly and accurately analyze seismic data in a field setting. Features include an interactive display of survey geometry, seismograms and simple processing tools. Expanding features to include field data from other methods (gravity, magnetics, DC/SP, GPR, and EM) as time permits.
			 
			\textbf{Geophysics Field Camp 2013} - Project to characterize the geothermal potential of Pagosa Springs, CO. Involved two weeks in the field gathering Gravity, Magnetics, EM, GPR, DC/SP, and Seismic data. This was followed by two weeks of data processing and technical report writing. Conclusions were published by the Colorado School of Mines Department of Geophysics.
			 
		   %\textbf{EPICS II: Geophysics Design} - Projects involving use of data analysis tools, such as R and JAVA, to interpret 					various data sets to solve a problem. Projects included analysis of sunspot data, correlation of local snowfall data, 					well-log analysis and finding \\ major driving forces for change in Lake Powell's water depth.
		   
		   %\textbf{EPICS I: Playground for the Disabled in South Africa} - Project \\involving research, conceptual design, graphical 				representation, subsystem analysis, materials list and budget of a playground for disabled children. Finalist in the 				EPICS 			competition.
		  %\end{itemize}

\section{OTHER \\ EXPERIENCE}
		  \begin{itemize}
		    \item Advanced Engineering Math Grader - Fall 2013.
		    \item Expected JAVA course Teaching Assistant - Spring 2014.
			\item High Grade Literary Journal Poetry Editor.			
			%\item Colorado School of Mines Varsity Swim Team.
			%\item Co-Founder and Officer of Mines Urban Gaming Club.
			\item Colorado School of Mines Peer Mentor.
			%\item Employment with the YMCA \& CSM Student Recreation Center.
		  \end{itemize}

\section{HONORS} 
	\begin{itemize}
	 \item Newmont Mining Scholar.
	 \item Naval Dolphin Scholarship Foundation Recipient.
	\end{itemize}
	 
%\section{{EXTRA-CURRICULAR \\ ACTIVITIES}} 
%	\begin{itemize}
%	 \item 
%	\end{itemize}
	 
\end{resume} 
\end{document}











\section{Appendices}


\end{document}