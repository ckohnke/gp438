\documentclass[a4paper,12pt]{texMemo}
\usepackage[english]{babel}
\usepackage{graphicx}
\usepackage{float}
\memoto{Prof. Dave Hale, Prof. T.K. Young}
\memofrom{Colton Kohnke}
\memodate{January 31, 2014}
\memosubject{GPGN438 Project Progress Report}

\begin{document}
\maketitle

This memo is to report on the monthly progress senior design project for GPGN438 titled ``Exploratory seismic data analysis in the field.'' \\

In the month of January, I have been researching ways to speed up the processes that display seismic data from memory. Part of this process finds the nearest gps point and then displays the seismic shot that is associated with that gps point. However, because there may not be data at every gps point, a binary search tree (BST) would be the most effective method because it is the most flexible when returning data. The other option is a hash map, but that method tends to be less flexible than a BST. \\

I have also been experimenting with displaying a range of shots based on a start shot number and end shot number. This will then be used to create sliders for the user to select a range of shots. It will also be used when the gui option to draw a selection queue to display a selection of specific shots. \\

I have also written pseudo-code for the gain slider. This part of the program will need to get what shot (or shot range) is selected, read the gain slider, apply the gain, and then display the gained section. \\

In the upcoming month I plan to start implementing and testing features such as reading observation reports, and creating the controls for gain and amplitude balancing. More research will be needed to determine the best way to implement these features. In the coming weeks, I also plan on starting to optimize the code so that it runs smoothly and effeciently.\\

\end{document}
