\documentclass[a4paper,11pt]{texMemo}
\usepackage[english]{babel}
\usepackage{graphicx, blindtext}
\memoto{Prof. Dave Hale, Prof. T.K. Young}
\memofrom{Colton Kohnke}
\memodate{September 13, 2013}
\memosubject{GPGN438 Project Initiation}

\begin{document}
\maketitle

This memo is to confirm the steps the have been accomplished towards starting a senior design project for GPGN438. \\

\begin{enumerate}
 \item I have registered for 1 (one) credit hour this semester with 2 (two) credit hours planned for the Spring. This will complete the 3-credit-hour requirement for GPGN438. 
 
 \item The title of my project is ``Exploratory Seismic Data Analysis in the Field.''
 
 \item The purpose of the project is to allow for an interactive interrogation of seismic data in a field setting. This software will allow for better survey decisions to be made in the field for seismic surveys. Currently, the software will be written in JAVA and the data from CSM Field Camp 2013 will be used for testing. A rough draft list of base features created by the client has been attached for reference with more to be added by me as the project progresses.
 
 \item I am currently a team of one, but increased perspective and manpower would be helpful for the later stages of the project. 
 
 \item My advisor and client are both Prof. Dave Hale.
 
 \item I have not yet developed a timeline and real budget for this project. Both of these items will be complete before the next progress report and will be attached to that report. 
 
 \item Professor Dave Hale is currently seeking shelter from the rain in Boulder, but I am planning on meeting with him next week (week of Sept. 16) in order to discuss next steps.
\end{enumerate}

Thank you!

\newpage

\section*{Rough Draft Project Features}
\begin{verbatim}
 Exploratory seismic data analysis in the field

Interactive display of survey geometry
---
get station (flag) locations
  from handheld GPS
  convert to UTM coordinates, if necessary
  store in spreadsheet or tab-delimited text
    station# stationXYZ

compute source and receiver locations
  for each valid shot FFID (unique field record identifier)
    use observer notes/files to determine 
      source station number (and offset/skid, if any)
      which recording channels are live
      mapping from live channel numbers to receiver station numbers
    use station locations to lookup UTM coordinates (x,y)
      for both source and receivers
    use USGS digital elevation maps to lookup elevations (z) 
    store in spreadsheet or tab text
       FFID SEGDfileName source station# channel# sourceXYZ receiverXYZ
  note that FFID increase sequentially
    but some FFIDs correspond to bad shots and must be ignored

displays source-receiver coordinates
  plot source/receiver/midpoint (x,y) in map view
  interactively specify a piecewise linear curve through these points
    begin by specifying just one line segment
    this curve defines the seismic "line"
  plot elevation profile source/receiver/midpoint z vs distance along curve
    project source/receiver/midpoint (x,y) onto curve
  use slider to select and show points for each FFID
    as we move the slider, the points (x,y) or z move along the line
    provides a graphical history of the seismic survey
    will help us catch mistakes

Interactive display of seismograms
---
convert seismograms in SEGD files to IEEE (float) files
  I have some Python code that does this
  should be translated to Java

display by FFID
  as we move the FFID slider
    display all seismograms for that FFID
      sorted by channel number
      sorted by signed receiver-source distance

display by midpoints within a circle
  as we move/resize the circle in map view of midpoints (x,y)
    display all seismograms for which midpoints lie within the circle
      sorted by signed receiver-source distance

display by offsets (signed receiver-source distances)
  as we move/resize a min-max slider for a range of offsets
    display all seismograms for which offset is within that range
      sorted by distance of midpoint along the line

a few interactive controls for 
  gain
  amplitude balancing
  zoom and scroll (already provided by edu.mines.jtk.mosaic)

Simple processing
---
surface wave attenuation
normal-moveout correction
...
\end{verbatim}

\end{document}